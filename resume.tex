% !TEX program = xelatex
\documentclass{resume}
\usepackage{zh_CN-Adobefonts_external} % Simplified Chinese Support using external fonts (./fonts/zh_CN-Adobe/)
\usepackage{lastpage}
\usepackage{fancyhdr}
\usepackage{linespacing_fix}
\renewcommand{\labelitemii}{$\circ$}

\begin{document}

\name{欧阳奎}
\basicInfo{
  % \phone{} \textperiodcentered\
  \email{2772509238@qq.com} \textperiodcentered\
  \github[github]{https://github.com/Mike-Smith-rem} \textperiodcentered\
  \homepage[rem.github.io]{https://github.com/Mike-Smith-rem}
}

\section{教育背景}
\datedsubsection{\textbf{北京航空航天大学}, 中国}{2019.09 -- 现在}
\role{本科,在读}{计算机科学与技术 GPA 3.6/4.0 排名 140/205}

% \section{科研经历}
% \datedsubsection{\textbf{北航系统结构研究所}}{2021.08 -- 现在}
% \begin{itemize}
%   \item 用排队论建模和修改 GPGPU-Sim 的流水线和片上网络结构,加速 GPU 仿真
%   \item 面向 GPU 统一虚拟内存的数据管理优化技术
%         \begin{itemize}
%           \item 修改 NVIDIA GPU 开源驱动,实现支持异构内存系统的 UVM 模块,能够在指定结点上分配页面
%           \item 识别 UVM 应用的数据管理瓶颈以及访存特征,实现异构内存下数据的优化放置
%         \end{itemize}
% \end{itemize}

\section{项目开发}
\subsection{\textbf{底层系统}}
\begin{itemize}
  % \item \textbf{Ayame 编译器} (Java, LLVM-IR, ARM):合作(主要负责寄存器分配与后端优化)
  %       \begin{itemize}
  %         \item SysY 编译器,使用 SSA IR,可导出 LLVM-IR/ARM32 汇编
  %         % \item 2021 华为毕昇杯一等奖,在近 1/3 的样例点上性能超过 \texttt{clang -O3}/\texttt{gcc -O3}
  %       \end{itemize}

  % \item \textbf{Racoon 编译器} (Rust, LLVM-IR):
  %       SysY - LLVM IR 编译器;北航软院 19 级编译课设 Rust 版参考实现

  \item \textbf{MIPS 处理器} (Verilog, MIPS):
        五级流水线 CPU,主要完成了MIPS的基础十条指令

  \item \textbf{MOS 操作系统} (C, MIPS):
        操作系统内核,包含了一个初步的 Shell
\end{itemize}

\subsection{\textbf{应用开发}}
\begin{itemize}
  % \item \textbf{HangGai} (SwiftUI, Vue, Rails):
  %       《航空航天概论》课程刷题应用,支持 Web 与 iOS,「航概」已上架 AppStore

  \item \textbf{FreeTalk} (Vue, Django):
        师生课程交流平台;2021年数据库大作业;合作(主要负责前端开发)

  \item \textbf{PaperDaily} (安卓程序App, Vue, Django):
        专家-企业需求对接小程序;合作(主要负责前端开发)
\end{itemize}

\section{获奖情况}
\subsection{\textbf{竞赛}}
\begin{itemize}
  \item \datedline{蓝桥杯软件类省赛(\textit{一等奖})}{2022.04}
  % \item \datedline{第 24 届CCF-CSP 软件能力认证(\textit{单次排名 1.51\%,累计排名 0.99\%})}{2021.09}
\end{itemize}

\subsection{\textbf{奖学金}}
\begin{itemize}
  \item \datedline{北航本科生学习优秀奖学金(\textit{二等奖})}{2020.11}
  \item \datedline{北航本科生国家励志奖学金}{2020.11}
  \item \datedline{北航本科生社会实践奖学金(\textit{一等奖})}{2020.11}
\end{itemize}

\section{专业技能}
\begin{itemize}
  \item \textbf{编程语言}:C, Java, Python, JavaScript, Ruby, Verilog

  % \item \textbf{编译/程序语言理论}:
  %   \begin{itemize}
  %     \item 熟悉 ANTLR 等解析器生成器,了解 SSA 和相关优化算法
  %     \item 了解函数式语言和类型系统,理解基于 Dependent Type 的形式化验证技术并学习过 Coq
  %   \end{itemize}

  \item \textbf{体系结构}:
    \begin{itemize}
      \item 熟悉流水线 CPU 原理,学习过 Verilog
    \end{itemize}

  \item \textbf{应用开发}:
    \begin{itemize}
      \item 技术栈:Web 前端 (Vue),App开发(uni-app)
    \end{itemize}

  % \item \textbf{开发环境}:常在 Linux 下使用 Emacs 与 JetBrains IDE;熟悉 Gitflow 协作流程
\end{itemize}

\section{其他}
\begin{itemize}
  % \item \textbf{助教工作}:
  %       \begin{itemize}
  %         \item \datedline{\textbf{程序设计基础训练}(北航信息类)}{2020.09 -- 2021.02}
  %               \role{课程助教}{负责出题和答疑}
  %         \item \datedline{\textbf{编译原理和技术}(北航软件学院)}{2021.09 -- 2022.01}
  %               \role{协助}{协助修改和完善指导书,并编写了课设参考实现}
  %         \item \datedline{\textbf{面向对象设计与构造}(北航计算机学院)}{2021.08 -- 2022.06}
  %               \role{课程助教及系统组}{参与课程系统(Rails + GitLab)的开发与运维,课程出题,答疑,跟班等}
  %       \end{itemize}
  \item \textbf{社会实践}:参加过挑战杯红旅赛道,获得社会实践先进单位称号
  % \item \textbf{博客}:已经在 \url{https://roife.github.io} 创作约 170 篇文章,月访问量最高逾 1.5k
  \item \textbf{外语}:英语(CET-6)
\end{itemize}
\end{document}
